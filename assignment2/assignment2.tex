\documentclass[11pt]{article}

\usepackage{graphicx}
\usepackage{fullpage}
\usepackage{url}
\usepackage{hyperref}
\usepackage{karnaugh-map}



\title{CS51 - Assignment 2\\Are you an introvert?\\{\normalsize Due: XXX at 11:59pm}}

\author{WRITE YOUR NAME HERE}
\date{\today}

\parindent=0in

\parskip 7.2pt 

\begin{document}
\maketitle

\section{Part 1: Disjunctive and Conjunctive Normal Forms}

% The following LaTex commands will be handy: \neg, \land, \lor

%TODO: Fill in the minterms and maxterms on the third and second ti last columns
\begin{displaymath}
\begin{array}{cccc|c|c|c}
A & B & C & D  & minterm & maxterm & I\\
\hline

0 & 0 & 0 & 0 & m_0 =  & M_0 =  & 0 \\
0 & 0 & 0 & 1 & m_1 =  & M_1 =  & 0 \\
0 & 0 & 1 & 0 & m_2 =  & M_2 =  & 0 \\
0 & 0 & 1 & 1 & m_3 =  & M_3 =  & 1 \\
0 & 1 & 0 & 0 & m_4 =  & M_4 =  & 0 \\
0 & 1 & 0 & 1 & m_5 =  & M_5 =  & 0 \\
0 & 1 & 1 & 0 & m_6 =  & M_6 =  & 1 \\
0 & 1 & 1 & 1 & m_7 =  & M_7 =  & 1 \\
1 & 0 & 0 & 0 & m_8 =  & M_8 =  & 0 \\
1 & 0 & 0 & 1 & m_9 =  & M_9 =  & 0 \\
1 & 0 & 1 & 0 & m_{10} =  & M_{10} =   & 1 \\
1 & 0 & 1 & 1 & m_{11} =  & M_{11} =  & 1 \\
1 & 1 & 0 & 0 & m_{12} =  & M_{12} =   & 1 \\
1 & 1 & 0 & 1 & m_{13} =  & M_{13} =  & 1 \\
1 & 1 & 1 & 0 & m_{14} =  & M_{14} = & 1 \\
1 & 1 & 1 & 1 & m_{15} = & M_{15} =   & 1
\end{array}
\end{displaymath}


%TODO: Calculate the DNF and CNF for the truth table above. Make sure you indicate which minterms and maxterms you used and then, as a second step, write the full expression using the four boolean variables. 
Disjunctive Normal Form: $I = $

Conjunctive Normal Form: $I = $

\section{Part 2: K-Maps}

% Option 1: You can work on pen and paper, snap a picture, upload it, and replace the following code after you uncomment it

% \begin{figure}[h]
%     \centering
%     \includegraphics[width=0.5\textwidth]{example-image} % Replace with your image filename
%     \caption{This is a sample figure caption.}
%     \label{fig:sample}
% \end{figure}

% Option 2: You can use the karnaugh-map LaTex package to create a pretty table like the one in the handout. Before you proceed, uncomment the following truth table. We have introduced a new column that assigns to each row a number m0 to m15 that corresponds to the 16 minterms. 

% \begin{displaymath}
% \begin{array}{cccc|c|c}
% A & B & C & D & I & minterm\\
% \hline
% 0 & 0 & 0 & 0 & 0 & m_0 \\
% 0 & 0 & 0 & 1 & 0 & m_1\\
% 0 & 0 & 1 & 0 & 0 & m_2\\
% 0 & 0 & 1 & 1 & 1 & m_3\\
% 0 & 1 & 0 & 0 & 0 & m_4\\
% 0 & 1 & 0 & 1 & 0 & m_5\\
% 0 & 1 & 1 & 0 & 1 & m_6\\
% 0 & 1 & 1 & 1 & 1 & m_7\\
% 1 & 0 & 0 & 0 & 0 & m_8\\
% 1 & 0 & 0 & 1 & 0 & m_9\\
% 1 & 0 & 1 & 0 & 1 & m_{10}\\
% 1 & 0 & 1 & 1 & 1 & m_{11}\\
% 1 & 1 & 0 & 0 & 1 & m_{12}\\
% 1 & 1 & 0 & 1 & 1 & m_{13}\\
% 1 & 1 & 1 & 0 & 1 & m_{14}\\
% 1 & 1 & 1 & 1 & 1 & m_{15}
% \end{array}
% \end{displaymath}


% Because Karnaugh maps work with Gray codes, the ordering within the cells is not sequential from $m_0$ to $m_{15}$$. Instead, if you uncomment and compile, you will see that certain cells are shifted. 

% \begin{karnaugh-map}[4][4][1][$D$][$C$][$B$][$A$]
%     \manualterms{m0, m1, m2, m3, m4, m5, m6, m7, m8, m9, m10, m11, m12, m13, m14, m15} % comment out this line when you are done editing minterms and maxterms
    % \minterms{} % fill in the cells you want to add a 1, separated by a comma. For example, \minterms{1} would add a 1 in the m1 cell. 
    % \maxterms{} % fill in the cells you want to add a 0. For example, \maxterms{5} would add a 0 in the m5 cell. 
    % \implicant{0}{5} % replace the cells you want to create a group around
    % \implicantedge{4}{12}{6}{14} % replace if you want a group to wrap around
% \end{karnaugh-map}


% Regardless of what option you chose, you need to provide the Boolean expression and explain how each term is derived by each colored group.

%\section{From your K-map to Python}

% Make sure you edit assignment2.py

%\section{Extra credit}

% Optionally, if you have chosen to do the extra credit, you should uncomment this section and work on your submission as in Part 2.



\end{document}
